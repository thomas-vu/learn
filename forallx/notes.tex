\documentclass[12pt, letterpaper, twoside]{article}
\usepackage{amssymb} \usepackage[utf8]{inputenc}

\newcommand*{\TTon}{\hspace{1.5em}\extracolsep{-1em}}
\newcommand*{\TToff}{\extracolsep{-1em}\hspace{.3em}}
\newcommand*{\TTbf}[1]{\textbf{\large #1}}
 
\title{forall x notes} \author{Thomas Vu} \date{\today}

\begin{document}

\maketitle
\newpage
\tableofcontents
\newpage

\section{Chapter 1 Practice Exercises}
 
\textbf{Part A} Which of the following are `sentences' in
the logical sense?
\begin{enumerate}
    \item \textbf{England is smaller than China.}
    \item \textbf{Greenland is south of Jerusalem.}
    \item Is New Jersey east of Wisconsin?
    \item \textbf{The atomic number of helium is 2.}
    \item \textbf{The atomic number of helium is $\pi$.}
    \item \textbf{I hate overcooked noodles.}
    \item Blech! Overcooked noodles!
    \item \textbf{Overcooked noodles are disgusting.}
    \item Take your time.
    \item \textbf{This is the last question.}
\end{enumerate}

\noindent \textbf{Part B} For each of the following: Is it a
tautology, a contradiction, or a contingent sentence?
\begin{enumerate}
    \item Caesar crossed the Rubicon. \textbf{(Contingent)}
    \item Someone once crossed the
      Rubicon. \textbf{(Contingent)}
    \item No one has ever crossed the
      Rubicon. \textbf{(Contingent)}
    \item If Caesar crossed the Rubicon, then someone
      has. \textbf{(Tautology)}
    \item Even though Caesar crossed the Rubicon, no one
      ever crossed the Rubicon. \textbf{(Contradiction)}
    \item If anyone has ever crossed the Rubicon, it was
      Caesar. \textbf{(Contingent)}
\end{enumerate}

\noindent \textbf{Part C} Look back at the sentences G1-G4
on p.11, and consider each of the following sets of
sentences. Which are consistent? Which are inconsistent?
\begin{enumerate}
    \item G2, G3, and G4 \textbf{(Consistent)}
    \item G1, G3, and G4 \textbf{(Inconsistent)}
    \item G1, G2, and G4 \textbf{(Consistent)}
    \item G1, G2, and G3 \textbf{(Consistent)}
\end{enumerate}

\noindent \textbf{Part D} Which of the following is
possible? If it is possible, give an example. If it is not
possible, explain why.
\begin{enumerate}
    \item A valid argument that has one false premise and
      one true premise

    This is possible. For example:
    \begin{enumerate}
        \item[] All men are carrots.
        \item[] Socrates is a man.
        \item[$\therefore$] Socrates is a carrot.
    \end{enumerate}

    \item A valid argument that has a false conclusion

    This is possible. The previous example for instance.

    \item A valid argument, the conclusion of which is a
      contradiction

    This is possible. For example:
    \begin{enumerate}
        \item[] It is both raining and not raining.
        \item[$\therefore$] It is both snowing and not
          snowing.
    \end{enumerate}

    \item An invalid argument, the conclusion of which is a
      tautology

    This is not possible. All invalid arguments have true
    premises and a false conclusion; this means the
    conclusion cannot be tautology (which is always true).

    \item A tautology that is contingent

    This is not possible since the definition of a
    contingent sentence requires that it not be a tautology.

    \item Two logically equivalent sentences, both of which
      are tautologies

    This is possible. In fact, any two tautologies will
    always be logically equivalent as they are both always
    true.

    \item Two logically equivalent sentences, one of which
      is a tautology and one of which is contingent

    This is not possible. Logical equivalence means that the
    sentences necessarily have the same truth-value. Since a
    contingent sentence may be false, it does not
    necessarily have the same truth value as a tautological
    sentence which is always true.

    \item Two logically equivalent sentences that together
      are an inconsistent set

    This is possible. Consider two sentences which are both
    contradictions. They must be logically equivalent since
    contradictions are always false; this also means it is
    not logically possible for the set containing these two
    sentences to be true at the same time.

    \item A consistent set of sentences that contains a
      contradiction

    This is not possible. Since this set contains a sentence
    which is always false, it is not logically possible for
    all the members of the set to be true at the same time.

    \item An inconsistent set of sentences that contains a
      tautology

    This is possible. Any inconsistent set of sentences will
    remain inconsistent if you add a tautology to it.
\end{enumerate}

\section{Chapter 2 Practice Exercises}
 
\textbf{Part A} Using the symbolization key given, translate
each English-language sentence into SL.
\begin{enumerate}
	\item[\textbf{M:}] Those creatures are men in suits.
	\item[\textbf{C:}] Those creatures are chimpanzees.
	\item[\textbf{G:}] Those creatures are gorillas.

	\item Those creatures are not men in suits.

	$\neg M$

	\item Those creatures are men in suits, or they are
          not.

	$M \vee \neg M$

	\item Those creatures are either gorillas or
          chimpanzees.

	$G \vee C$

	\item Those creatures are neither gorillas nor
          chimpanzees.

	$\neg (G \vee C)$

	\item If those creatures are chimpanzees, then they
          are neither gorillas nor men in suits.

	$C \rightarrow \neg (G \vee M)$

	\item Unless those creatures are men in suits, they
          are either chimpanzees or they are gorillas.

	$M \vee (C \vee G)$
\end{enumerate}

\noindent \textbf{Part B} Using the symbolization key given,
translate each English-language sentence into SL.
\begin{enumerate}
	\item[\textbf{A:}] Mister Ace was murdered.
	\item[\textbf{B:}] The butler did it.
	\item[\textbf{C:}] The cook did it.
	\item[\textbf{D:}] The Duchess is lying.
	\item[\textbf{E:}] Mister Edge was murdered.
	\item[\textbf{F:}] The murder weapon was a frying
          pan.
	
	\item Either Mister Ace or Mister Edge was murdered.

	$A \vee E$

	\item If Mister Ace was murdered, then the cook did
          it.

	$A \rightarrow C$

	\item If Mister Edge was murdered, then the cook did
          not do it.

	$E \rightarrow \neg C$

	\item Either the butler did it, or the Duchess is
          lying.

	$B \vee D$

	\item The cook did it only if the Duchess is lying.

	$C \rightarrow D$

	\item If the murder weapon was a frying pan, then
          the culprit must have been the cook.

	$F \rightarrow C$

	\item If the murder weapon was not a frying pan,
          then the culprit was either the cook or the
          butler.

	$\neg F \rightarrow (C \vee B)$

	\item Mister Ace was murdered if and only if Mister
          Edge was not murdered.

	$A \leftrightarrow \neg E$

	\item The Duchess is lying, unless it was Mister
          Edge who was murdered.

	$D \vee E$

	\item If Mister Ace was murdered, he was done in
          with a frying pan.

	$A \rightarrow F$

	\item Since the cook did it, the butler did not.

	$C \wedge (C \rightarrow \neg B)$

	\item Of course the Duchess is lying!

	$D$
\end{enumerate}

\noindent \textbf{Part C} Using the symbolization key given,
translate each English-language sentence into SL.
\begin{enumerate}
	\item[\textbf{E$_1$:}] Ava is an electrician.
	\item[\textbf{E$_2$:}] Harrison is an electrician.
	\item[\textbf{F$_1$:}] Ava is a firefighter.
	\item[\textbf{F$_2$:}] Harrison is a firefighter.
	\item[\textbf{S$_1$:}] Ava is satisfied with her
          career.
	\item[\textbf{S$_2$:}] Harrison is satisfied with
          his career.
	
	\item Ava and Harrison are both electricians.

	$E_1 \wedge E_2$

	\item If Ava is a firefighter, then she is satisfied
          with her career.

	$F_1 \rightarrow S_1$

	\item Ava is a firefighter, unless she is an
          electrician.

	$F_1 \vee E_1$

	\item Harrison is an unsatisfied electrician.

	$E_2 \wedge \neg S_2$

	\item Neither Ava nor Harrison is an electrician.

	$\neg (E_1 \vee E_2)$

	\item Both Ava and Harrison are electricians, but
          neither of them find it satisfying.

	$E_1 \wedge \neg S_1 \wedge E_2 \wedge \neg S_2$

	\item Harrison is satisfied only if he is a
          firefighter.

	$S_2 \rightarrow F_2$

	\item If Ava is not an electrician, then neither is
          Harrison, but if she is, then he is too.

	$E_1 \leftrightarrow E_2$

	\item Ava is satisfied with her career if and only
          if Harrison is not satisfied with his.

	$S_1 \leftrightarrow \neg S_2$

	\item If Harrison is both an electrician and a
          firefighter, then he must be satisfied with his
          work.

	$(E_2 \wedge F_2) \rightarrow S_2$

	\item It cannot be that Harrison is both an
          electrician and a firefighter.

	$\neg (E_2 \wedge F_2)$

	\item Harrison and Ava are both firefighters if and
          only if neither of them is an electrician.

	$(F_2 \wedge F_1) \leftrightarrow (\neg E_2 \wedge
          \neg E_1)$
\end{enumerate}

\noindent \textbf{Part D} Give a symbolization key and
symbolize the following sentences in SL.
\begin{enumerate}
	\item[\textbf{A:}] Alice is a spy.
	\item[\textbf{B:}] Bob is a spy.
	\item[\textbf{C:}] The code has been broken.
	\item[\textbf{G:}] The German embassy will be in an
          uproar.

	\item Alice and Bob are both spies.

	$A \wedge B$

	\item If either Alice or Bob is a spy, then the code
          has been broken.

	$(A \vee B) \rightarrow C$

	\item If neither Alice nor Bob is a spy, then the
          code remains unbroken.

	$\neg (A \vee B) \rightarrow \neg C$

	\item The German embassy will be in an uproar,
          unless someone has broken the code.

	$G \vee C$

	\item Either the code has been broken or it has not,
          but the German embassy will be in an uproar
          regardless.

	$(C \vee \neg C) \wedge G$

	\item Either Alice or Bob is a spy, but not both.

	$(A \vee B) \wedge \neg (A \wedge B)$
\end{enumerate}

\noindent \textbf{Part E} Give a symbolization key and
symbolize the following sentences in SL.
\begin{enumerate}
	\item[\textbf{G:}] Gregor plays first base.
	\item[\textbf{L:}] The team will lose.
	\item[\textbf{C:}] Gregor's mom will bake cookies.
	\item[\textbf{M:}] There is a miracle.

	\item If Gregor plays first base, then the team will
          lose.

	$G \rightarrow L$

	\item The team will lose unless there is a miracle.

	$L \vee M$

	\item The team will either lose or it won't, but
          Gregor will play first base regardless.

	$(L \vee \neg L) \wedge G$

	\item Gregor's mom will bake cookies if and only if
          Gregor plays first base.

	$C \leftrightarrow G$

	\item If there is a miracle, then Gregor's mom will
          not bake cookies.

	$M \rightarrow \neg C$
\end{enumerate}

\noindent \textbf{Part F} For each argument, write a
symbolization key and translate the argument as well as
possible into SL.
\begin{enumerate}
	\item If Dorothy plays the piano in the morning,
          then Roger wakes up cranky. Dorothy plays piano in
          the morning unless she is distracted. So if Roger
          does not wake up cranky, then Dorothy must be
          distracted.

	\item[\textbf{P:}] Dorothy plays piano in the
          morning.
	\item[\textbf{R:}] Roger wakes up cranky.
	\item[\textbf{D:}] Dorothy is distracted.

    \begin{enumerate}
        \item[] $P \rightarrow C$
        \item[] $P \vee D$
        \item[$\therefore$] $\neg C \rightarrow D$
    \end{enumerate}

	\item It will either rain or snow on Tuesday. If it
          rains, Neville will be sad. If it snows, Neville
          will be cold. Therefore, Neville will either be
          sad or cold on Tuesday.

	\item[\textbf{R:}] It will rain on Tuesday.
	\item[\textbf{S:}] It will snow on Tuesday.
	\item[\textbf{N$_1$:}] Neville will be sad on
          Tuesday.
	\item[\textbf{N$_2$:}] Neville will be cold on
          Tuesday.
	
    \begin{enumerate}
        \item[] $R \vee S$
        \item[] $R \rightarrow N_1$
        \item[] $S \rightarrow N_2$
        \item[$\therefore$] $N_1 \vee N_2$
    \end{enumerate}

	\item If Zoog remembered to do his chores, then
          things are clean but not neat. If he forgot, then
          things are neat but not clean. Therefore, things
          are either neat or clean--- but not both.

	\item[\textbf{Z:}] Zoog remembered to do his chores.
	\item[\textbf{C:}] Things are clean.
	\item[\textbf{N}] Things are neat.
	
    \begin{enumerate}
        \item[] $Z \rightarrow (C \wedge \neg N)$
        \item[] $\neg Z \rightarrow (N \wedge \neg C)$
        \item[$\therefore$] $(N \vee C) \wedge \neg (N
          \wedge C)$
    \end{enumerate}
\end{enumerate}

\noindent \textbf{Part G} For each of the following: (a) Is
it a wff of SL? (b) Is it a sentence of SL, allowing for
notational conventions?
\begin{enumerate}
	\item (a) No. (b) No.
	\item (a) No. (b) Yes.
	\item (a) Yes. (b) Yes.
	\item (a) No. (b) No.
	\item (a) Yes. (b) Yes.
	\item (a) No. (b) No.
	\item (a) No. (b) Yes.
	\item (a) No. (b) Yes.
	\item (a) No. (b) No.
\end{enumerate}

\noindent \textbf{Part H}
\begin{enumerate}
	\item Are there any wffs of SL that contain no
          sentence letters? Why or why not?

	No, since the base case of a wff is always a
        sentence letter.

	\item In the chapter, we symbolized an
          \emph{exclusive or} using $\vee$, $\wedge$, and
          $\neg$. How could you translate an \emph{exclusive
            or} using only two connectives? Is there any way
          to translate an \emph{exclusive or} using only one
          connective?

	3 connectives: $(A \vee B) \wedge \neg (A \wedge B)$

	2 connectives: ?

	1 connective: ?
\end{enumerate}

\section{Chapter 3 Practice Exercises}

\noindent \textbf{Part A} Determine whether each sentence is
a tautology, a contradiction, or a contingent
sentence. Justify your answer with a complete or partial
truth table where appropriate.
\begin{enumerate}
	\item $A \rightarrow A$ \textbf{(Tautology)}
	
	\begin{center}
	\begin{tabular}{c|@{\TTon}*{3}{c}@{\TToff}}
	$A$ & $A$ & $\rightarrow$ & $A$ \\ \hline 1 & 1 &
          \TTbf{1} & 1 \\ 0 & 0 & \TTbf{1} & 0
	\end{tabular}
	\end{center}
	
	\item $\neg B \wedge B$ \textbf{(Contradiction)}
	
	\begin{center}
	\begin{tabular}{c|@{\TTon}*{4}{c}@{\TToff}}
	$B$ & $\neg$ & $B$ & $\wedge$ & $B$ \\ \hline 1 & 0
          & 1 & \TTbf{0} & 1 \\ 0 & 1 & 0 & \TTbf{0} & 0
	\end{tabular}
	\end{center}
	
	\item $C \rightarrow\neg C$ \textbf{(Contingent)}
	
	\begin{center}
	\begin{tabular}{c|@{\TTon}*{4}{c}@{\TToff}}
	$C$ & $C$ & $\rightarrow$ & $\neg$ & $C$ \\ \hline 1
          & 1 & \TTbf{0} & 0 & 1 \\ 0 & 0 & \TTbf{1} & 1 & 0
	\end{tabular}
	\end{center}
	
	\item $\neg D \vee D$ \textbf{(Tautology)}
	
	\begin{center}
	\begin{tabular}{c|@{\TTon}*{4}{c}@{\TToff}}
	$D$ & $\neg$ & $D$ & $\vee$ & $D$ \\ \hline 1 & 0 &
          1 & \TTbf{1} & 1 \\ 0 & 1 & 0 & \TTbf{1} & 0
	\end{tabular}
	\end{center}
	
	\item $(A \leftrightarrow B) \leftrightarrow
          \neg(A\leftrightarrow \neg B)$
          \textbf{(Tautology)}
	
	\begin{center}
	\begin{tabular}{c|c|@{\TTon}*{9}{c}@{\TToff}}
	$A$ & $B$ & $(A$ & $\leftrightarrow$ & $B)$ &
          $\leftrightarrow$ & $\neg$ & $(A$ &
          $\leftrightarrow$ & $\neg$ & $B)$ \\ \hline 1 & 1
          & 1 & 1 & 1 & \TTbf{1} & 1 & 1 & 0 & 0 & 1 \\ 1 &
          0 & 1 & 0 & 0 & \TTbf{1} & 0 & 1 & 1 & 1 & 0 \\ 0
          & 1 & 0 & 0 & 1 & \TTbf{1} & 0 & 0 & 1 & 0 & 1
          \\ 0 & 0 & 0 & 1 & 0 & \TTbf{1} & 1 & 0 & 0 & 1 &
          0
	\end{tabular}
	\end{center}
	
	\item $(A\wedge B) \vee (B\wedge A)$
	\item $(A \rightarrow B) \vee (B \rightarrow A)$
	\item $\neg[A \rightarrow (B \rightarrow A)]$
	\item $(A \wedge B) \rightarrow (B \vee A)$
	\item $A \leftrightarrow [A \rightarrow (B \wedge
          \neg B)]$
	\item $\neg(A \vee B) \leftrightarrow (\neg A \wedge
          \neg B)$
	\item $\neg(A\wedge B) \leftrightarrow A$
	\item $\bigl[(A\wedge B) \wedge\neg(A\wedge B)\bigr]
          \wedge C$
	\item $A\rightarrow(B\vee C)$
	\item $[(A \wedge B) \wedge C] \rightarrow B$
	\item $(A \wedge\neg A) \rightarrow (B \vee C)$
	\item $\neg\bigl[(C\vee A) \vee B\bigr]$
	\item $(B\wedge D) \leftrightarrow [A
          \leftrightarrow(A \vee C)]$
\end{enumerate}

\noindent \textbf{Part B} Determine whether each pair of
sentences is logically equivalent. Justify your answer with
a complete or partial truth table where appropriate.
\begin{enumerate}
	\item $A$, $\neg A$
	\item $A$, $A \vee A$
	\item $A\rightarrow A$, $A \leftrightarrow A$
	\item $A \vee \neg B$, $A\rightarrow B$
	\item $A \wedge \neg A$, $\neg B \leftrightarrow B$
	\item $\neg(A \wedge B)$, $\neg A \vee \neg B$
	\item $\neg(A \rightarrow B)$, $\neg A \rightarrow
          \neg B$
	\item $(A \rightarrow B)$, $(\neg B \rightarrow \neg
          A)$
	\item $[(A \vee B) \vee C]$, $[A \vee (B \vee C)]$
	\item $[(A \vee B) \wedge C]$, $[A \vee (B \wedge
          C)]$
\end{enumerate}

\noindent \textbf{Part E} Answer each of the questions below
and justify your answer.
\begin{enumerate}
	\item Suppose that $\mathcal{A}$ and $\mathcal{B}$
          are logically equivalent. What can you say about
          $\mathcal{A}\leftrightarrow\mathcal{B}$?

	$\mathcal{A}\leftrightarrow\mathcal{B}$ is a
          tautology.  Every line of the complete truth table
          will be true since $\leftrightarrow$ evaluates to
          true when its arguments have the same truth value.

	\item Suppose that
          $(\mathcal{A}\wedge\mathcal{B})\rightarrow\mathcal{C}$
          is contingent. What can you say about the argument
          ``$\mathcal{A}, \mathcal{B},
          \therefore\mathcal{C}$''?

	The argument is invalid. Since
        $(\mathcal{A}\wedge\mathcal{B})\rightarrow\mathcal{C}$
        is contingent, there is some line where
        $(\mathcal{A}\wedge\mathcal{B})$ is true but
        $\mathcal{C}$ is false. This means there is some
        line where the premises $\mathcal{A}$ and
        $\mathcal{B}$ are true, and the conclusion
        $\mathcal{C}$ is false. Therefore, the argument is
        invalid.

	\item Suppose that $\{\mathcal{A},\mathcal{B},
          \mathcal{C}\}$ is inconsistent. What can you say
          about
          $(\mathcal{A}\wedge\mathcal{B}\wedge\mathcal{C})$?

	$(\mathcal{A}\wedge\mathcal{B}\wedge\mathcal{C})$ is
          a contradiction. Since the set
          $\{\mathcal{A},\mathcal{B}, \mathcal{C}\}$ is
          inconsistent, there is no line of a complete truth
          table where all of $\mathcal{A}$, $\mathcal{B}$,
          $\mathcal{C}$ are true. This means that
          $(\mathcal{A}\wedge\mathcal{B}\wedge\mathcal{C})$
          will be false on every line, or in other words, a
          contradiction.

	\item Suppose that $\mathcal{A}$ is a
          contradiction. What can you say about the argument
          ``$\mathcal{A}, \mathcal{B},
          \therefore\mathcal{C}$''?

	The argument is valid. There cannot be a line where
        both premises $\mathcal{A}$ and $\mathcal{B}$ are
        true, yet $\mathcal{C}$ is false, which is required
        for an invalid argument. This is because
        $\mathcal{A}$ is a contradiction and not true on any
        line.

	\item Suppose that $\mathcal{C}$ is a
          tautology. What can you say about the argument
          ``$\mathcal{A}, \mathcal{B},
          \therefore\mathcal{C}$''?

	The argument is valid. There cannot be a line where
        both premises $\mathcal{A}$ and $\mathcal{B}$ are
        true, yet $\mathcal{C}$ is false, which is required
        for an invalid argument. This is because
        $\mathcal{C}$ is a tautology and true on every line.

	\item Suppose that $\mathcal{A}$ and $\mathcal{B}$
          are logically equivalent. What can you say about
          $(\mathcal{A}\vee\mathcal{B})$?

	Not much. $\mathcal{A}$ and $\mathcal{B}$ are either
        both true or both false on some line, so
        $(\mathcal{A}\vee\mathcal{B})$ could either be true
        or false on that line as well.
        $(\mathcal{A}\vee\mathcal{B})$ could be a tautology,
        a contradiction, or contingent.

	\item Suppose that $\mathcal{A}$ and $\mathcal{B}$
          are \emph{not} logically equivalent. What can you
          say about $(\mathcal{A}\vee\mathcal{B})$?

	$(\mathcal{A}\vee\mathcal{B})$ cannot be a
          contradiction, i.e.  it is either a tautology or
          contingent. Since $\mathcal{A}$ is not logically
          equivalent to $\mathcal{B}$, there must be some
          line where $\mathcal{A}$ differs in truth value
          from $\mathcal{B}$, which means
          $(\mathcal{A}\vee\mathcal{B})$ must be true on
          some line.
\end{enumerate}

\end{document}
