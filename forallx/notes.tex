\documentclass[12pt, letterpaper, twoside]{article}
\usepackage{amssymb}
\usepackage[utf8]{inputenc}
 
\title{forall x notes}
\author{Thomas Vu}
\date{\today}

\begin{document}

\maketitle
\newpage
\tableofcontents
\newpage

\section{Chapter 1 Practice Exercises}
 
\textbf{Part A} Which of the following are 'sentences' in
the logical sense?
 
\begin{enumerate}
    \item \textbf{England is smaller than China.}
    \item \textbf{Greenland is south of Jerusalem.}
    \item Is New Jersey east of Wisconsin?
    \item \textbf{The atomic number of helium is 2.}
    \item \textbf{The atomic number of helium is $\pi$.}
    \item \textbf{I hate overcooked noodles.}
    \item Blech! Overcooked noodles!
    \item \textbf{Overcooked noodles are disgusting.}
    \item Take your time.
    \item \textbf{This is the last question.}
\end{enumerate}

\noindent \textbf{Part B} For each of the following: Is it a
tautology, a contradiction, or a contingent sentence?

\begin{enumerate}
    \item Caesar crossed the Rubicon. \textbf{(Contingent)}
    \item Someone once crossed the Rubicon. \textbf{(Contingent)}
    \item No one has ever crossed the Rubicon. \textbf{(Contingent)}
    \item If Caesar crossed the Rubicon, then someone
      has. \textbf{(Tautology)}
    \item Even though Caesar crossed the Rubicon, no one
      ever crossed the Rubicon. \textbf{(Contradiction)}
    \item If anyone has ever crossed the Rubicon, it was
      Caesar. \textbf{(Contingent)}
\end{enumerate}

\noindent \textbf{Part C} Look back at the sentences G1-G4
on p.11, and consider each of the following sets of
sentences. Which are consistent? Which are inconsistent?

\begin{enumerate}
    \item G2, G3, and G4 \textbf{(Consistent)}
    \item G1, G3, and G4 \textbf{(Inconsistent)}
    \item G1, G2, and G4 \textbf{(Consistent)}
    \item G1, G2, and G3 \textbf{(Consistent)}
\end{enumerate}

\noindent \textbf{Part D} Which of the following is
possible? If it is possible, give an example. If it is not
possible, explain why.

\begin{enumerate}
    \item A valid argument that has one false premise and
      one true premise

    This is possible. For example:
    \begin{itemize}
        \item[] All men are carrots.
        \item[] Socrates is a man.
        \item[$\therefore$] Socrates is a carrot.
    \end{itemize}

    \item A valid argument that has a false conclusion

    This is possible. The previous example for instance.

    \item A valid argument, the conclusion of which is a
      contradiction

    This is possible. For example:
    \begin{itemize}
        \item[] All men are carrots.
        \item[$\therefore$] It is both raining and not raining.
    \end{itemize}

    \item An invalid argument, the conclusion of which is a
      tautology

    This is not possible. All invalid arguments have true
    premises and a false conclusion; this means the
    conclusion cannot be tautology (which is always true).

    \item A tautology that is contingent

    This is not possible since the definition of a
    contingent sentence requires that it not be a tautology.

    \item Two logically equivalent sentences, both of which
      are tautologies

    This is possible. In fact, any two tautologies will
    always be logically equivalent as they are always true.

    \item Two logically equivalent sentences, one of which
      is a tautology and one of which is contingent

    This is not possible. Logical equivalence means that the
    sentences necessarily have the same truth-value. Since a
    contingent sentence may be false, it does not
    necessarily have the same truth value as a tautological
    sentence which is always true.

    \item Two logically equivalent sentences that together
      are an inconsistent set

    This is possible. Consider two sentences which are both
    contradictions. They must be logically equivalent since
    contradictions are always false; this also means it is
    not logically possible for the set containing these two
    sentences to be true at the same time.

    \item A consistent set of sentences that contains a
      contradiction

    This is not possible. Since this set contains a sentence
    which is always false, it is not logically possible for
    all the members of the set to be true at the same time.

    \item An inconsistent set of sentences that contains a
      tautology

    This is possible. Any inconsistent set of sentences will
    remain inconsistent if you add a tautology to it.

\end{enumerate}

\section{Chapter 2 Practice Exercises}
 
\end{document}
